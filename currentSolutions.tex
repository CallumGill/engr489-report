\chapter{Background}
This chapter explores current solutions that a company could use for managing company information and policies and the informing of staff. The main aspects that we will be exploring are the organization of information, the presentation and navigation of this information for users and the ease of updating this information and informing users of these changes that they may need to be made aware of. We will explore two main solution types, knowledge management systems and learning management systems.

\section{Knowledge Management Systems}
Knowledge management systems main goal is to provide effective and efficient knowledge management. This includes the need to archive information and provide an interface for easily retrieving and display this information. They often take the form of collaboratively driven systems known as Wikis. Wikis provide an environment for conversational knowledge to be created and iteratively improved in an ad-hoc manor and `combine the best elements of earlier conversational knowledge management technologies, while avoiding many of their disadvantages' \cite{maier2011knowledge, leuf2001wiki}.

\subsection{Wikis}
A Wiki refers to is a set of linked web pages that are created by incremental changes applied by a group of collaborating users and the software that is used to manage the pages and information. The key characteristics of a wiki are:

\begin{itemize}
\item It enables web documents to be authored collectively.
\item It uses a simple markup scheme (usually a simplified version of HTML, although HTML is frequently permitted).
\item Wiki content is not reviewed by any editor or coordinating body prior to its publication.
\item New web pages are created when users create a hyperlink that points nowhere.
\end{itemize}

From these characteristics you can see that Wikis are a collaborative system that relies on group contribution to remain up to date and the creation of new content. This creates a large knowledge base that can be updated easily and frequently with  an active community. However because of the lack of moderation outside of community amendments, some of the published knowledge can be incorrect. This is contrary to what is required within a company for documents such as policies and other content that requires an approval process.When Wikis have been used as company KMS they often exhibit multiple issues these have been identified as \cite{kiniti2013wikis}:

\begin{itemize}
\item Lack of a clear purpose for the wiki. Wikis are often created by individuals or small team and are then adopted by other teams or individuals.
\item Usability. These issues range from poor hosting infrastructure, user knowledge of syntax, duplicated data and validity of data.
\item Integrating the wiki into established work practices. 
\item Role of management. Usage of Wikis is gated by management involvement, encouragement, training and rewarding wiki users all contribute to a more successful wiki.
\item Social issues. A critical mass of users is required for a successful wiki.
\item Organizational Culture. Users are required to be open to sharing knowledge.
\end{itemize}

\section{Learning Management Systems}
LMS serve three main purposes, to manage users (both learners and facilitators), track online learning and progress and keep records of activity. LMS are often split into two sub categories, corporate or education, each focusing on a different learning environment. These share many characteristics and there is often overlap between the categories. Common features of a LMS include \cite{mcintosh2014vendors}:

\begin{itemize}
\item Online Courses
\item Classroom management
\item Communication and collaboration tools
\item Content management and development tools
\item Assessment and testing
\item Reporting
\end{itemize}

Each LMS offers its own suite of functionality, because of this they can fit a multitude of roles in numerous different environments. But the end goal of most them is to create and manage a collaborative online or blended learning environment. One of the most popular LMS's that many others are based on or extend is Moodle.

\subsection{Moodle}
Moodle is an open-source free LMS that is distributed under the GNU public license. Moodle is overall designed to promote community-oriented learning methodology, involving collaborative activities and active participation from its users. This is aimed at creating an online learning community. Moodle includes a multitude of community features such as, forums, blogs, announcements and wikis. It also includes learning management and tracking features such as tests and quizzes, assignments, report, student profiles and activity tracking \cite{mcintosh2014vendors, muhsen2013moodle}. However because of the multitude of features, like many LMS it requires training to use and manage effectively. Also for some of it applications it is over engineered and many of its features go unused or used improperly/poorly. This is a factor with broad LMS systems, because they target a broad range of applications, they lack the clarity of how to best use their features or they some of the features are adapted in incorrect or inefficient ways.

\section{Summary}
There are many KMS and LMS systems that can be used for managing and teaching information for employees. However, neither of them have them as their primary use. KMS are used to store and retrieve this information but have no designated tracking of learning or directed learning environment. While LMS are over engineered towards learning, are focused on a collaborative learning experience and the testing or validation of the users learning rather than a source for easy to navigate information that is targeted towards the user automatically.